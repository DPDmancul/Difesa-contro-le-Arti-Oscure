%! TeX root = ../Parte2.tex

\section{Blocco dei tracker}

\begin{myframe}{Blocco tracker pubblicitari}
  Spesso gli annunci pubblicitari ci tracciano: sanno che pagine visitiamo\\ e quanto ci soffermiamo in determinati punti.

  \medskip\pause
  uBlock Origin blocca gli annunci pubblicitari sulle pagine web\\ e molti altri elementi traccianti.
\end{myframe}


\begin{myframe}{Blocco tracker su Firefox}
  Firefox fornisce di default molte impostazioni per il blocco degli elementi traccianti.

  \includegraphics[width=.7\textwidth]{img/firefox_tracker}
\end{myframe}


\begin{myframe}{Firefox containers}
  Inoltre attivando i containers si possono creare dei \textbf{profili} da usare per \textbf{siti diversi}.

  Così da non condividere i cookie traccianti, ma anche permettendo l'uso di account diversi senza sloggarsi.

  \smallskip
  \includegraphics[width=.7\textwidth]{img/firefox_containers}
\end{myframe}

\begin{myframe}{Facebook container}
  Esiste un container particolare chiamato ``Facebook container'' che reindirizza in automatico tutte le pagine di Meta a lui.

  Protegge anche dai pulsanti ``Mi piace'' presenti nelle pagine web disabilitandoli.

  \smallskip
  \includegraphics[width=.8\textwidth]{img/facebook_container}
\end{myframe}

\begin{myframe}{Altre estensioni}
  \begin{description}
    \item[HTTPS everywhere / Firefox HTTPS-only] Reindirizza alla versione https dei siti se disponibile;
    \item[Decentraleyes] Elimina chiamate a CDN;
    \item[Privacy badger] Blocca molti tracker.
  \end{description}
\end{myframe}

\begin{myframe}{DNS}
\end{myframe}

