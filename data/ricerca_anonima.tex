%! TeX root = ../Parte2.tex

\section{Ricerca anonima}

\begin{myframe}{Problemi dei motori di ricerca traccianti}
  \begin{itemize}[<+->]
    \item Google è il motore di ricerca più usato e\only<1>\dots\only<+->{ il più grande servizio di pubblicità online;}
    \item Le ricerche vengono usate per capire i nostri interessi;
    \item I risultati di ricerca vengono scelti e ordinati in base ai nostri interessi, per il loro interesse (effetto bolla).
  \end{itemize}
\end{myframe}

\begin{myframe}{Motori di ricerca non traccianti}
  \begin{itemize}
    \item DuckDuckGo
    \item Searx (\url{paulgo.io}, \url{searx.be})
    \item Metager
  \end{itemize}
\end{myframe}

\begin{myframe}{Altre funzionalità di Metager e DuckDuckGo}
  I bang permettono di fare velocemente ricerche su altri siti
  \begin{itemize}[<+->]
    \item \texttt{!w Privacy} -- Cerca ``Privacy'' su Wikipedia
    \item \texttt{!yt Corso di LaTeX} -- Cerca su YouTube
    \item \texttt{!wrenit duck} -- Dizionario inglese-italiano
    \item \texttt{!wa int log(x) dx} -- Wolfram alpha
    \item \texttt{!bang} -- Elenco tutti i bang
  \end{itemize}
\end{myframe}


