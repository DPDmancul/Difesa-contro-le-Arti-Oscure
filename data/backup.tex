%! TeX root = ../Parte3.tex

\section{Backup}

\begin{myframe}{Perdita dei dati}
  Una volta che i dati sono persi c'è poco da fare.

  Esistono metodi per \textbf{recupero dati}, ma funzionano bene \textbf{raramente}.

  La cosa migliore è la \textbf{prevenzione}: il backup!
\end{myframe}

\begin{myframe}{La regola 3-2-1}
  Quanti backup fare?
  \pause
  \begin{itemize}[<+->]
    \item Almeno \textbf{3 copie},
    \item \textbf{2} copie su \textbf{supporti diversi},
    \item \textbf{una} copia in un \textbf{luogo diverso}.
  \end{itemize}

  \medskip
  \only<+->{\textbf{Importante!} Verifica sempre che il backup sia stato fatto correttamente.}
\end{myframe}

\begin{myframe}{Cosa non è un backup}
  \begin{itemize}[<+->]
    \item Dropbox o altri cloud;
    \item RAID,
    \item Altri servizi di sincronizzazione.
  \end{itemize}
  \medskip
  \only<+->{Questi infatti \textbf{non proteggono} da \textbf{corruzione} dei dati,\\ \textbf{cancellazioni} involontarie e \textbf{malware}.}
\end{myframe}

\begin{myframe}{Backup incrementale}
  Tenere \textbf{traccia dei cambiamenti} per
  \begin{itemize}
    \item Diminuire tempi backup,
    \item Ripristinare vecchi scenari.
  \end{itemize}
\end{myframe}

\begin{myframe}{Software per backup}
  \begin{description}
    \item[GNU/Linux] Déjà Dup, Borg, \dots
    \item[Windows] Backup e ripristino
    \item[MacOS] Time Machine
  \end{description}
  \pause\bigskip

  \textbf{Importante!}
  \begin{itemize}[<+->]
    \item Verificate di eseguire il backup e non solo salvare il progetto.
    \item Periodicamente controllate che i backup possano essere ripristinati correttamente.
  \end{itemize}
\end{myframe}

\begin{myframe}{Backup configurazione}
  Oltre al backup dei file può essere utile il backup della configurazione (impostazioni, pacchetti installati, \dots).
  \begin{description}[<+->]
    \item[dotfiles] Backup file di configurazione. Spesso condivisi su GitLab, Codeberg, SourceHut, GitHub, \dots
    \item[Timeshift] Backup pacchetti e impostazioni. Può sfruttare snapshot BTRFS.
    \item[MxLinux snapshot] Crea un'immagine avviabile del sistema corrente.
    \item[Nix, NixOs, Guix] Gestore dei pacchetti (Nix; funziona anche su MacOS) e distribuzioni GNU/Linux (NixOS, Guix) interamente configurabili da un singolo file. Supportano rollback all'avvio.
  \end{description}
\end{myframe}

\begin{myframe}{Backup del cloud}
  Non abbiamo \textbf{nessuna garanzia} i nostri dati sul cloud siano per sempre:\\bisogna fare un backup anche di quelli.

  \pause\bigskip\bigskip

  Se invece volgiamo un backup di un contenuto pubblico su internet\\possiamo usare the Wayback Machine.

  Molto utile anche per citare siti web che potrebbero sparire o essere modificati.
\end{myframe}

\begin{myframe}{Recupero in extremis}
  Se non riusciamo più ad accendere il computer e abbiamo dati che non sono stati salvati su un backup si può tentare un ripristino con una \textbf{live GNU/Linux}:
  \pause
  \begin{enumerate}[<+->]
    \item Creare una chiavetta live con una distro GNU/Linux;
    \item Avviare da chiavetta sul computer fuori uso;
    \item Col file manager copiare i file dal disco del computer su un disco esterno.
  \end{enumerate}

  \bigskip
  \only<+->{La stessa procedura può essere usata da \textbf{malintenzionati} per spiare, ma anche per cambiare file (e configurazioni nel sistema): per questo è importante criptare il disco.}
\end{myframe}

\begin{myframe}{Diversi tipi di formattazione}
  \begin{description}
    \item[Rapida (default)] Cancella mappatura dei blocchi (l'``indice''), ma non i dati.\\È possibile recuperare file con buona probabilità.
    \item[A basso livello] Cancella tutto, anche i dati. È quasi impossibile recuperare i file.
  \end{description}
\end{myframe}


