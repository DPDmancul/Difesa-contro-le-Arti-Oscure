%! TeX root = ../Parte3.tex

\section{Malware}

\begin{myframe}{Cosa sono i malware}
  \textbf{Software malevolo} che nuoce all'utente.

  \pause
  Ce ne sono moltissimi tipi, ecco i più  famosi:
  \begin{description}
    \item[Virus e worm] Software malevolo in grado di replicarsi su altri dispositivi.
    \item[Trojan horse] Software utile che contiene funzionalità dannose nascoste al suo interno.
    \item[Backdoor] Accesso non autorizzato al sistema.
    \item[Spyware e keylogger] Ruba dati e credenziali degli utenti.
    \item[Hijacker] Causano apertura di pagine web indesiderate.
    \item[Rabbit] Esauriscono le risorse del dispositivo.
    \item[Adware e Malvertising] Si diffondono attraverso le pubblicità.
    \item[Ransomware] Cripta i dati presenti sul disco pretendendo un riscatto.
  \end{description}
\end{myframe}

\begin{myframe}{Come difendersi dai malware}
  \begin{itemize}[<+->]
    \item Prestando \textbf{attenzione} ai siti visitati, ai programmi avviati, alle chiavette inserite, \dots
    \item Con la scansione periodica degli \textbf{antivirus} (specialmente su Windows e Android)\\
    es: ClamAV, ClamWin, LibreAV, Bitdefender, \dots
    \item Tenendo \textbf{aggiornati} i programmi e il sistema operativo: spesso i malware sfruttano i bug dei programmi.
  \end{itemize}
\end{myframe}

\begin{myframe}{Scansione da altri dispositivi}
  \begin{itemize}[<+->]
    \item Molti antivirus offrono una live ISO (da mettere su una chiavetta USB) per scansionare il computer senza accendere il sistema operativo;
    \item Alcuni siti (come Virus total) offrono un servizio remoto di scansione dei file con più antivirus;
    \item Ci sono estensioni del browser che permettono i blocco di siti poco affidabili.
  \end{itemize}
\end{myframe}

\begin{myframe}{Open source}
  Il software open source è più sicuro di quello closed source perché:
  \begin{itemize}[<+->]
    \item Ci si può accertare \textbf{non sia un trojan} e che \textbf{non abbia backdoor o spyware}, eventualmente si possono rimuovere;
    \item È \textbf{revisionato} da molte più persone;
    \item \textbf{Patch di sicurezza} rilasciate più velocemente e disponibili a tutti;
  \end{itemize}
\end{myframe}

