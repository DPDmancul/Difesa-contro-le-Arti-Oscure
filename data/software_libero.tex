%! TeX root = ../Parte3.tex

\section{Software libero}

\begin{myframe}{La libertà degli utenti}
  tecnicamente tutto il \textbf{software proprietario} è malware\\perché nuoce alla libertà degli utenti.

  \pause\bigskip

  {\Large\textbf{4 libertà essenziali}}
  \pause
   \begin{description}[<+->]
     \item[Libertà 0] Eseguire il programma come si desidera, per qualsiasi scopo;
     \item[Libertà 1] Studiare il funzionamento del programma e modificarlo perché funzioni a piacimento; (Necessario sia open source)
     \item[Libertà 2] Redistribuire il programma per aiutare gli altri;
     \item[Libertà 3] Distribuire le versioni modificate del programma, per il beneficio di tutti.
   \end{description}
\end{myframe}

\begin{myframe}{Le licenze}%\footnotesize
  \begin{tabular}{c|cccc}
    & FOSS & Esempi licenze & Esempi licenze\\
    & & software & per lavori creativi\\\hline
    Pubblico dominio & Sì & Public domain, CC0 & Public domain, CC0\\\hline
    Licenze permissive &  Sì & Apache 2.0, MIT, & CC-BY\\
    (obbligo citare autore)&  & 3-Clause BSD, MPL\\\hline
    \textbf{Copyleft} & Sì & GPL/AGPL/LGPL 2 e 3 & CC-BY-SA\\
    (obbligo ridistribuire\\sotto stessa licenza)\\\hline
    Licenze non commerciali & No & & CC-BY-NC-SA\\\hline
    Licenze proprietarie & No & EULA & Copyright \\\hline
    Segreto industriale & No & - & -
  \end{tabular}
\end{myframe}

\begin{myframe}{Software (e non solo) liberi famosi}
  \begin{itemize}[<+->]
    \item GNU/Linux, BSD
    \item Firefox, IceCat, TorBrowser, Chromium
    \item VLC media player
    \item Wikipedia, Wikimedia
    \item \LaTeX
    \item Libreoffice
    \item Open Street Map
    \item Signal, Element (Matrix)
    \item glibc, gcc, GNU R, Python
    \item GIMP, Blender
  \end{itemize}
\end{myframe}

