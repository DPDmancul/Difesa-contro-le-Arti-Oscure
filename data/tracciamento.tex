%! TeX root = ../Parte2.tex

\section{Tracciamento}

\begin{myframe}{Baratto e informazione}
  Il sistema economico globale è ``\textbf{baratto} pro'': i \textbf{soldi} sono merce di scambio universale, usati per \textbf{ripagare la sottrazione} di un bene o di tempo.

  \bigskip\pause
  Questo modello cozza con la società delle informazioni.

  \bigskip\pause
  \begin{columns}
    \onslide<+->{
      \begin{column}{.4\textwidth}\centering
        \begin{tikzpicture}
          \node (mela1) {Mela};
          \node (pera1) [right=of mela1] {Pera};
          \onslide<+->{
            \node[align=center] (pera2) [below=of mela1] {Pera\\};
            \node[align=center] (mela2) [below=of pera1] {Mela\\};
            \draw[->] (mela1) -- (mela2);
            \draw[->] (pera1) -- (pera2);
          }
        \end{tikzpicture}
      }
    \end{column}
    \begin{column}{.4\textwidth}\centering
      \onslide<+->{
        \begin{tikzpicture}
          \node (mela1) {Mela};
          \node (idea1) [right=of mela1] {Idea};
          \onslide<+->{
            \node[align=center] (idea2) [below=of mela1] {Idea\\};
            \node[align=center] (mela2) [below=of idea1] {Idea\\Mela};
            \draw[->] (mela1) -- (mela2);
            \draw[->] (idea1) -- (idea2);
          }
        \end{tikzpicture}
      }
    \end{column}
  \end{columns}
\end{myframe}

\begin{myframe}{I costi del gratis}
  Le \textbf{informazioni} non si consumano: \textbf{non} sono \textbf{barattabili}.

  Inoltre possono essere \textbf{copiate a costo nullo}.

  \medskip\pause
  Una copia di una mela non può essere usufruita, una \textbf{copia} dell'informazione è l'informazione stessa e \textbf{può essere usufruita}.

  \pause
  Per \textbf{essere ripagati} del lavoro svolto a creare l'informazione\\bisogna \textbf{trovare un modo di barattarla}.

  Per questo \textbf{paghiamo per copie} dell'informazione\\molto \textbf{più di quanto costi la copia}.

  \medskip\pause
  Quando \textbf{non paghiamo} o qualcuno paga per noi (sponsor)\\o \textbf{paghiamo senza saperlo}.

  Su internet spesso \textbf{la merce siamo noi}: i servizi ``gratis'' più diffusi li paghiamo con noi stessi barattandoli per \textbf{le nostre informazioni personali} e riservate.
\end{myframe}

\begin{myframe}{Le nostre tracce}
  I \textbf{siti} che visitiamo e le \textbf{app} che usiamo \textbf{memorizzano} le nostre attività.

  \pause\medskip
  Grazie a queste informazioni possono \textbf{profilarci}.

  \pause\medskip
  Il nostro profilo può essere usato per:
  \begin{itemize}[<+->]
    \item pubblicità mirata
    \item prezzi personalizzati (non a nostro favore ovviamente)
    \item proibirci accesso ad alcuni servizi
    \item molto altro \dots
  \end{itemize}

\end{myframe}

\begin{myframe}{Chi ci traccia?}
  Alcuni esempi \pause
  \begin{itemize}[<+->]
    \item \textbf{I motori di ricerca} Salvano i dati delle nostre ricerche
    \item \textbf{I social network} Memorizzano tutti i nostri dati e i nostri post (anche quelli privati e le bozze);
    \item \textbf{Le app di messaggistica e mail} Memorizzano le nostre chiamate e i nostri messaggi cercando le parole che usiamo di più per identificare i nostri interessi;
    \item \textbf{Servizi cloud} Memorizzano i nostri dati sui loro computer, dove rimangono anche dopo che noi li abbiamo cancellati;
    \item \textbf{Phishing} Mail o siti che fingendo di essere siti ufficiali di banche, enti statali, \dots{} convincono la gente a fornire informazioni personali come dati riservati, password, codici di carte di credito, \dots;
    \item \textbf{Virus} Rubano i dati memorizzati nel nostro computer e nel nostro telefono esponendo così tutti i nostri dati personali.
  \end{itemize}
\end{myframe}

\begin{myframe}{Cookie}
  I cookie \textbf{identificano} un browser su un specifico dispositivo, collegandolo all'utente.

  Usati per salvare gli account evitando di fare il login ad ogni pagina che si visita.

  \pause\medskip
  Vengono anche usati per tenere traccia delle pagine che visitiamo: la pubblicità Google e i pulsanti ``Mi piace'' di Facebook usano gli stessi cookie del motore di ricerca e del social in ogni pagina dove compaiono.

  Per questo il GDPR obbliga i fornitori dei servizi online a mostrare la richiesta di accettazione dei cookie.
\end{myframe}
